\documentclass[10pt,conference,compsocconf]{IEEEtran}

\usepackage{hyperref}
\usepackage{graphicx}	% For figure environment


\begin{document}
\title{CS433-Machine Learning Project 1}

\author{
  Amaury Combes - Vincenzo Bazzucchi - Alexis Montavon\\
}

\maketitle

\begin{abstract}
  The Higgs Boson Kaggle challenge was put in place by physicists in CERN in order to analyze the massive data gathered during their research with the Large Hadron Collider. The idea was to use the best algorithms to predict if a particle collision event was a signal of the Higgs Boson. This challenge was actually one of the biggest ever on Kaggle and we reproduced it in our Machine Learning class at EPFL.
\end{abstract}

\section{Introduction}

TODO: at the end

\section{Model and Methods}
\label{sec:model}
TODO: explain step chosen (what works, failed, why, using tables)
\subsection{Preprocessing}
After diving into the dataset the first question that came to our attention was what to do with the undefined values \textit{-999.0}. Two options came to mind, setting them to 0 or to the average of every valid values in each feature. We opted for the second option as it seemed more coherent. This is done by the \textit{mean\_spec} function in the \textit{run.py} file.

\subsection{Models}


\section{Results}

TODO: I guess best results we got and the exact technics and parameters, give exact loss (mean of cross validation maybe)

\section{Summary}

TODO: Retrace best option we used in short

\bibliographystyle{IEEEtran}
\bibliography{literature}

\end{document}
